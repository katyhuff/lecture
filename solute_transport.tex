\section{Mechanisms of Solute Transport}

Solutes in groundwater are transported due to various mechanisms.

\subsection{Diffusion}
\label{ss:diffusion}

Solutes will move from a location of high concentration to a location of low 
concentration due to the random thermal kinetic enery of the solute. Brownian 
motion is responsible for diffusion.

\begin{figure}[htbp!]
  \begin{center}
    \def\svgwidth{.6\textwidth}
    \input{Diffusion.eps_tex}
  \end{center}
  \caption{In diffusive transport, particles move according to random thermal 
  motion down the concentration gradient \cite{jrpol_diffusion.svg_2011}.}
  \label{fig:Diffusion}
\end{figure}

A mathematical description of this phenomenon can be addressed by describing the 
particle's motion as a random walk and applying statistical analysis. Another 
derivation of this phenomenon describes the flux, $J$ through a plane in a 
volume, and gives Fick's Law for porous media,

\begin{align}
  J_{xdif} &= -D' \frac{dC}{dx}\nonumber\\
  \label{ficks}
  \intertext{where}
  J_x &= \mbox{ Mass flux through the y-z plane }[kg/m^2/s]\nonumber\\
  D' &= \mbox{ Effective diffusion coefficient }[m^2/s]\nonumber\\
  C &= \mbox{ Concentration }[kg/m^3].\nonumber\\
\end{align}

The determination of the diffusion coefficient can be empirical or analytical.  
The Arrhenius equation gives a thermal dependence of the diffusion coefficient 
as does the Stokes-Einstein relationship.

The effecive diffusion coefficient for porous media is dependent on porosity and  
tortuosity, which affect the diffusion behavior of a solute. In a saturated porous 
medium, diffusion takes place only within the pore volume where liquid is 
present. To take this into account, the effective diffusion coefficient in a 
porous medium is reduced by both the absence of fluid in the matrix volume, and 
by the so-called ``tortuosity'' of the pathways. 

\begin{align}
  J_{xdif} &= -n\tau D_m \frac{dC}{dx}\\
  \label{ficks}
  \intertext{where}
  J_{xdif} &= \mbox{ Mass flux through the y-z plane in a porous medium}[kg/m^2/s]\nonumber\\
  n &= \mbox{ porosity }[\%]\nonumber\\
  \tau &= \mbox{ tortuosity }[-]\nonumber\\
  D_m &= \mbox{ Molecular diffusion coefficient }[m^2/s]\nonumber\\
  C &= \mbox{ Concentration }[kg/m^3].\nonumber
\end{align}

The tortuosity of a saturated porous medium can be estimated from the porosity 
with an equation developed by Millington and Quirk (1961),

\begin{align}
  \tau &= \frac{n_w^{7/3}}{n^2} \label{millington}
  \intertext{which for a saturated medium is estimated as }
  \tau &= n^{1/3}.  \\
  % Below is true, but where to put it . . . ?
  % D' = n\tau D_m = nD^{\star}
\end{align}


\subsection{Dispersion \cite{schwartz_fundamentals_2003}}

Solutes can be driven to move due to mechanical mixing, known as dispersion.  
Solutes acting under dispersion may travel further than they would due to 
advection alone. During fluid mixing in which a fluid with one solute 
concentration displaces a fluid with another concentration, dispersion causes a 
zone of mixing to develop around the advective front.

Dispersion in groundwater is caused by a combination of two phenomena. The first 
is mechanical dispersion, in which local variability of the fluid velocity 
around the mean due to mechanical heterogeneities of the rock at many scales.  
The second phenomenon is diffusion, discussed in section \ref{ss:diffusion}.

A mathematical description of this phenomenon will come generally from fluid 
dynamics. The total mechanical dispersive and diffusive phenomena can be 
combined as a total dispersive expression,
\begin{align}
  D &= D_{mdis} + \tau D_m
  \intertext{where}
  D_{mdis}&= \mbox{ coefficient of mechanical dispersivity }[m^2/s]\nonumber\\
          &=\alpha_{L} \frac{v_x^2}{|v|} + \alpha_{TH} \frac{v_y^2}{|v|} + 
          \alpha_{TV} \frac{v_z^2}{|v|},\\ \alpha_L  &= \mbox{ longitudinal 
          dispersivity }[m],\nonumber\\
  \alpha_{TH}  &= \mbox{ horizontal transverse dispersivity }[m],\nonumber\\
  \alpha_{TV}  &= \mbox{ vertical transverse dispersivity }[m],\nonumber\\
  \tau &= \mbox{ tortuosity }[-],\nonumber
  \intertext{and}
  D_m &= \mbox{ coefficient of molecular diffusion }[m^2/s]\nonumber.
  \label{totaldisp}
\end{align}
The dispersive mass flux can therefore be described, as in Burnett and Frind, 
1987,
\begin{align}
  J_{dis} &= J_{mdis} + J_{dif} \\
          &= -n\left(D_{mdis} + \tau D_m\right)\nabla C\\
          &= -nD\nabla C
  \label{dispflux}
  \intertext{where}
  D &= \left[ \begin{array}{ccc}
                D_{xx}  & D_{xy}  & D_{xz}  \\
                D_{yx}  & D_{yy}  & D_{yz}  \\
                D_{zx}  & D_{zy}  & D_{zz}  \end{array} \right],\nonumber\\
  D_{ij} &=
         \begin{cases} \alpha_{L} \frac{v_x^2}{|v|} + \alpha_{TH} 
           \frac{v_y^2}{|v|} + \alpha_{TV} \frac{v_z^2}{|v|} &  i=j=x,\\
                       \alpha_{L} \frac{v_y^2}{|v|} + \alpha_{TH} 
                       \frac{v_x^2}{|v|} + \alpha_{TV} \frac{v_z^2}{|v|} 
                       + \tau  D_m & i=j=y,\\
                       \alpha_{L} \frac{v_z^2}{|v|} + \alpha_{TV} 
                       \frac{v_y^2}{|v|} + \alpha_{TV} \frac{v_x^2}{|v|}  
                       + \tau D_m & i=j=z,\\
                       \left(\alpha_L - \alpha_{TH}\right)\frac{v_xv_y}{|v|} 
                       + \tau D_m & i=x, j=y,\\
                       \left(\alpha_L - \alpha_{TV}\right)\frac{v_xv_z}{|v|} & 
                       i=(x,z),j=(z,x),\\
                       \left(\alpha_L - \alpha_{TV}\right)\frac{v_yv_z}{|v|} & 
                       i=(y,z), j=(z,y),\\
         \end{cases}
\intertext{and}
|v| &= \sqrt{v_x^2 +v_y^2 + v_x^2}.
\end{align}
For uniform flow, where $v=v_x$ and $v_y=v_z=0$,
\begin{align}
  D_x &= D_L \nonumber\\
      &= \alpha_L v_x + \tau D_m\\
  D_y &= D_{TH} \nonumber\\
      &= \alpha_{TH} v_x + \tau D_m\\
  D_z &= D_{TV} \nonumber\\
      &= \alpha_{TV} v_x + \tau D_m.
  \label{uniflow}
\end{align}

\subsection{Advection}

Solutes are transported by advection when they move along with the water in 
which they are dissolved. The expression for mass flux due to dispersion can be 
added to the expression for mass flux due to advection to give total mass flux 
due to both,

\begin{align}
  J &=  J_{mdis} + J_{dif} + J_{adv}\\
    &=  J_{dis} + J_{adv}\\
    &= -nD\nabla C + nvC\\
    &=\left(-nD_{xx} \frac{\partial C}{\partial x}
            -nD_{xy} \frac{\partial C}{\partial y}
            -nD_{xz} \frac{\partial C}{\partial z}
             + nv_xC \right)\hat{\imath} \nonumber\\
    & + \left(-nD_{yx} \frac{\partial C}{\partial x}
            -nD_{yy} \frac{\partial C}{\partial y}
            -nD_{yz} \frac{\partial C}{\partial z}
            + nv_yC \right)\hat{\jmath} \nonumber\\
    & + \left(-nD_{zx} \frac{\partial C}{\partial x}
            -nD_{zy} \frac{\partial C}{\partial y}
            -nD_{zz} \frac{\partial C}{\partial z}
            + nv_zC \right)\hat{k}.
            \label{jadvdisp}
\end{align}

For unidirectional flow

\begin{align}
  J &=\left(-nD_{xx} \frac{\partial C}{\partial x}
             + nv_xC \right)\hat{\imath}
             + \left( -nD_{yy} \frac{\partial C}{\partial y}
            \right)\hat{\jmath}
            + \left( -nD_{zz} \frac{\partial C}{\partial z}
            \right)\hat{k}.
            \label{jadvdisp}
\end{align}


\section{Nuclide Model Implementations}

\subsection{Buffer Model}

The buffer component is a long, but finite, cylindrical geometry with a series 
of cylindrical waste packages within it. These waste packages can be modeled as  
a single line source, a series of cylinders, or a series of points. If they are  
modeled as a single line source, the flux emitted by the line boundary will be 
expressed, in one dimension, as
\begin{align}
  vC_0 &= v\sum_{p=1}^{p=P} C_p
  \intertext{where}
  vC_0 &= \mbox{ constant solute flux during the timestep }[kg/m^2/s ]\nonumber\\
  C_p &= \mbox{ available concentration at waste package p }[kg/m^3]\nonumber\\
  P &= \mbox{ number of waste packages within the buffer cylinder }[\#].\nonumber\\
  \label{wpsum}
\end{align}

If the line source is chosen, the one dimensional models in cases I, II, or III 
can be used, as can the lumped parameter model.  In these cases, the flux at the 
inner buffer radius is described by the cylindrical source waste form. The 
concentration gradient can be defined as zero either at the outer buffer radius, 
or at infinity. These approximate solutions may not give significantly different 
results for many situations, but each will be investigated.

If alternatively, the waste packages are modeled as discrete cylinders, the 
buffer must also be modeled as a discrete cylinder within the broader, far field  
semi-infinite medium. 

\subsection{Far Field Solution}

The three dimensional solution with a cylindrical source of constant flux is the
most appropriate treatment for the boundary conditions of the far field 
component. With the length of the tunnels and outer radial dimension of the 
buffer, the far field can be modeled as a semi-infinite medium and the flux at 
the location of the perceived aquifer can be queried.  In case IV, the flux 
escaping the cylindrical buffer will be an output of the buffer model.  The 
concentration gradient is zero at infinity.

The far field may also be modeled as a lumped parameter system. It may either be  
modeled as a single lumped parameter component or as a series of smaller 
dimension lumped parameter components. For a homogeneous medium, the difference  
in accuracy between these two approaches will be zero, tho

\section{Groundwater Hydrology
\cite{schwartz_fundamentals_2003, wang_introduction_1982}}

Groundwater hydrology is the study of groundwater movement.

\subsection{Hydraulic Head }

Hydraulic head is the hydrologic indicator of potential energy. It has units of 
length, specifically height above a geological \emph{datum}, the zero point 
relative to which heads are measured. 

One way to derive this notion is according to Hubbart's derivation 
\cite{wang_introduction_1982}. The work to raise the water to a pressure, $P$, 
and the work required to raise the water to an elevation, $z$, are additive.

The work required to raise a mass, $m$, to a pressure, $P$, is described by the 
integral
\begin{align}
  W_P &= \frac{1}{m}\int_0^PVdP
  \intertext{where}
  m  &= \mbox{ mass of the water }[kg]\nonumber\\
  V  &= \mbox{ volume of the water }[m^3]\nonumber\\
  P  &= \mbox{ pressure }[ kg/m/s^2].\nonumber
\end{align}

In an incompressible fluid such as water, the work $W_P$ required to raise a 
unit mass
to some pressure, $P$, becomes
\begin{align}
  W_P &= P/\rho_w \label{workp}
  \intertext{where}
  \rho_w  &= \mbox{ the density of water}[kg/m^3].\nonumber
\end{align}

The work required to raise a unit mass to elevation $z$ from elevation $z_{ref}$  
(the `datum') is simply the work, $W_g$ against gravity
\begin{align}
  W_g &= g(z-z_{ref})
  \label{workg}
  \intertext{where}
  g  &= \mbox{ acceleration of gravity }[m/s^2].\nonumber
\end{align}

From these expressions for work the potential energy function, $\phi$, of the 
two  separate potentials, pressure and elevation, act additively on a unit mass 
of groundwater. Their combined effects can be written as the sum,
\begin{align}
  \phi &= W_P + W_g\nonumber\\
       &= \frac{P}{\rho_w} + g(z-z_{ref}.)
  \label{potentialfunc}
  \intertext{If}
  z_{ref} &= 0\nonumber
  \intertext{and the potential energy is directly proportional to gravity}
  \phi &= gh \nonumber
  \intertext{where}
  h &= \mbox{ Darcy's experimental head }[m],\nonumber
  \intertext{then}
  h &= \frac{\phi}{g}\nonumber\\
    &= \frac{P}{\rho_wg} + z\nonumber\\
    &=  \psi + z \label{bernoulli}
  \intertext{where}
  z &= \mbox{ elevation head }[m]\nonumber\\
  \psi &= \mbox{ pressure head }[m].\nonumber
\end{align}

Thus, the total hydraulic head, $h$, of a column of water is the vertical 
distance from the datum ($h=0$) to the height of the water surface.  Elevation 
head, $z$, is the distance from the datum to the measuring point in the flow 
field. The pressure head, $\psi$, is the pressure, $P$ in $[Pa]$, at that 
measurement point, divided by the density of the fluid, $\rho_w$, and the 
accelation of gravity, $g$. Pressure head, elevation head, and total head are 
shown in Figure
\ref{fig:head}. 

\begin{figure}[htbp!]
  \begin{center}
    \def\svgwidth{.7\textwidth}
    \input{head.eps_tex}
    \caption{Various types of head value in a column of water in a well are 
    measured with respect to the datum.}
    \label{fig:head}
  \end{center}
\end{figure}


Different lithologies and candidate sites have different head gradients. Due to 
anisotropies in their hydraulic conductivity tensors (see section 
\ref{subsec:cond}), some lithologies, such as clay, are less likely to have 
strong head gradients. 

\subsection{Porosity}

Porosity is a hydraulic parameter which significantly contributes to the 
hydraulic conductivity of a medium. 

The porosity of a medium is the ration of void volume, $V_v$, in the medium to 
the total volume, $V_T$ of the medium. As shown in Figure \ref{fig:wetDry}, the 
volume of water sufficient to saturate a packed sand matrix is equivalent to the 
void volume.

\begin{figure}[htbp!]
  \begin{center}
    \def\svgwidth{\textwidth}
    \input{wetDry.eps_tex}
  \end{center}
  \caption{The ratio of the volume of saturation water to the total packed sand 
  volume gives the porosity, $n$, in $[\%]$ \cite{heath_basic_1983}. }
  \label{fig:wetDry}
\end{figure}

The porosity of a material is therefore mathematically defined as \begin{align}
  n_T &= \frac{\mbox{void volume}}{\mbox{total volume}}\nonumber\\
      &= \frac{V_v}{V_T}\nonumber\\
      &= \frac{V_T-V_s}{V_T}
  \label{porosity}
  \intertext{where}
  n_T &= \mbox{ total porosity }[\%]\nonumber\\
  V_v &= \mbox{ void volume }[m^3]\nonumber\\
  V_T &= \mbox{ total volume }[m^3]\nonumber\\
  V_s &= \mbox{ volume of solids }[m^3].\nonumber
  \intertext{The dry solid volume can be defined as}
  V_s    &= \frac{m_s}{\rho_s}
  \intertext{and the total dry volume can be defined as}
  V_T    &= \frac{m_s}{\rho_b},
  \intertext{where}
  \rho_s &= \mbox{ grain density, or density of solids }[kg/m^3]\nonumber\\
  m_s    &= \mbox{ mass of solids }[kg]\nonumber\\
  \intertext{and}
  \rho_b &= \mbox{ bulk (dry) density }[kg/m^3],\nonumber\\
  \intertext{equation \eqref{porosity} can be rewritten}
  n_T    &= 1 - \frac{\rho_b}{\rho_s}.
\end{align}

The total porosity is a combination of primary and secondary porosity, shown in 
Figure \ref{fig:porosity}. Primary porosity is the macroscopically homogeneous  
void space present between grains in a geological matrix. This type of porosity 
includes empty space between grains of sand or consolidated rock characteristic 
of the level of packing or method of consolidation of the matrix. Secondary 
porosity is not intrinsic
to the rock, but induced in it. Examples of secondary porosity include features  
such as fractures, lava tubes, and caverns.

\begin{figure}[htbp!]
  \begin{center}
    \def\svgwidth{.8\textwidth}
    \input{porosity.eps_tex}
  \end{center}
  \caption{Primary porosity, a homogeneous characteristic of the rock matrix, 
  and secondary porosity, such as discrete fractures, caverns, and lava tubes,
  both contribute to total porosity \cite{heath_basic_1983}.} 
  \label{fig:porosity}
\end{figure}

Equipped with a notion of the total porosity, it is possible to define the 
effective porosity shown in figure \ref{fig:effPorosity}, which is the 
interconnected porosity through which fluid may flow, \begin{align}
  n_{eff} &= \frac{V_c}{V_T}n_T.  \label{effPorosity}
  \intertext{where}
  V_c &= \mbox{ interconnected volume. }\nonumber
\end{align}
Effective porosity is more influential on ground water flow in a porous medium 
than the total porosity, as it contributes to the permeability of the medium.  
In some media, the effective porosity is significantly different than the total  
porosity, resulting in much lower flow rates than the total porosity alone would 
indicate. Granite, for example, has an effective porosity that is nearly three 
magnitudes lower than its total porosity. 

\vspace{1cm}
\begin{figure}[htbp!]
  \begin{center}
    \def\svgwidth{.5\textwidth}
    \input{effPorosity.eps_tex}
  \end{center}
  \caption{A medium with a poorly connected pore volume has low effective 
  porosity, and vice versa.}
  \label{fig:effPorosity}
\end{figure}

\subsection{Darcy's Law}

Darcy's Law is analgous to Fourier's Law of heat conduction, and states that 
flow occurs as a function of a head gradient with a speed that obeys the 
hydraulic properties of the medium. Darcy's law for one dimensional flow is 

\begin{align}
  \frac{Q}{A} &= -K\frac{h_2 - h_1}{l_2-l_1} \label{darcy1d}
  \intertext{where}
  Q &= \mbox{ volumetric flow rate }[m^3/s]\nonumber
  \intertext{and}
  A &= \mbox{ cross sectional area of flow field }[m^2]\nonumber\\
  K &= \mbox{ hydraulic conductivity }[ m/s ].\nonumber\\
  h_i &= \mbox{ hydraulic head measured at point i } [m]\nonumber\\
  l_i &= \mbox{ linear position of point i } [m]\nonumber
\end{align}

Using the the definition of specific discharge (also known as Darcy flux), $q = 
Q/A$, and expressing the law in its differential form gives

\begin{align}
  q &= -K\frac{\partial h}{\partial l}.
  \label{darcy1ddiff}
\end{align}
  
The three dimensional form requires that the one dimensional form in equation 
\eqref{darcy1ddiff} be true for each spatial component, such that

\begin{align}
  \vec{q} &= -K\nabla{h}
  \label{darcy3d}
  \intertext{where}
  \vec{q} &= \mbox{ specific discharge vector }[m^3/s]\nonumber\\
  &= q_x\hat{\imath} + q_y\hat{\jmath} + q_z\hat{k}\nonumber
  \intertext{and}
  \nabla{h} &= \frac{\partial h}{\partial x} + \frac{\partial h}{\partial y} + 
  \frac{\partial h}{\partial z}. \nonumber
\end{align}

In the case of anisotropic media, the hydraulic conductivity, $K$ is given as a 
spatially heterogenous tensor, $\textbf{K}$. The general, three dimensional, 
anisotropic Darcy equation is therefore 

\begin{align}
  \vec{q} &= -\textbf{K}\nabla{h}
  \label{darcyanisotropic}
  \intertext{where}
  \textbf{K} &= \mbox{ hydraulic conductivity tensor }[ m/s ].\nonumber
\end{align}

\subsection{Hydraulic Conductivity}
\label{subsec:cond}

The hydraulic conductivity of a medium is a tensor that describes the ease with 
which water passes through the pore spaces of the medium in all directions.  
This tensor may be homogeneous or heterogeneous and isotropic or anisotropic. 

\subsubsection{Heterogeneity of the Hydraulic Conductivity}

Heterogeneity of a hydrogeological unit describes the spatial variability of the 
hydraulic conductivity. If the hydraulic conductivity is the same when
measured at all points in the hydrogeological medium,

\begin{align}
  \forall i,j \in V : K(x_i,y_i,z_i) = K(x_j,y_j,z_j),
  \label{homogeneous}
\end{align}
then the medium is homogenous. If instead, the hydraulic conductivity is
different at different locations, such that 
\begin{align}
  \exists i\ne j \in V : K(x_i,y_i,z_i) \ne K(x_j,y_j,z_j),
  \label{heteroeneous}
\end{align}

then the medium is heterogeneous.

\subsubsection{Anisotropy of the Hydraulic Conductivity}

Anisotropy of a hydrogeological unit describes the local directional dependence 
of the hydraulic conductivity tensor. Specifically, at a single location, an 
isotropic medium will demonstrate the same hydraulic conductivity in all 
directions, while an anisotropic medium will not.

Mathematically speaking, if the conductivity is constant at all measurement 
angles, $\theta$, such that

\begin{align}
  \forall \theta_i,\theta_j \in (0,2\pi) : K(\theta_i,x,y,z) = K(\theta_j,x,y,z) 
  \\
  \label{isotropic}
\end{align}
then the medium is isotropic. If instead, the angular components of the
hydraulic conductivity are nonconstant, such that
\begin{align}
  \exists \theta_i\ne\theta_j \in (0,2\pi) : K(\theta_i,x,y,z) \ne 
  K(\theta_j,x,y,z),
  \label{anisotropic}
\end{align}
then the medium is anisotropic and the tensor is expressed most generally,
\begin{align}
  \textbf{K} &= \mbox{ hydraulic conductivity tensor }[ m/s ]\nonumber\\
             &= \left[ \begin{array}{ccc}
                K_{xx}  & K_{xy}  & K_{xz}  \\
                K_{yx}  & K_{yy}  & K_{yz}  \\
                K_{zx}  & K_{zy}  & K_{zz}  \end{array} \right]
    \label{condfull}
    \intertext{where}
    K_{ij} &=\mbox{jth component in the i direction}.\nonumber
\end{align}
The general tensor in equation \eqref{condfull} is a second rank symmetric 
tensor and can always be simplified by aligning the tensor with the principal 
axis of anisotropy\cite{schwartz_fundamentals_2003}. This principal direction is 
defined by the directions of maximum ($K_{\parallel}$), minimum ($K_{\perp}$), 
and intermediate ($K_{\parallel}\times K_{\perp}$) hydraulic conductivity. That 
is, by aligning the tensor with the flow axis, it may be reduced to a diagonal 
matrix,

\begin{align}
    \textbf{K} &= \left[ \begin{array}{ccc}
                 K_{xx}  & 0       & 0  \\
                 0       & K_{yy}  & 0  \\
                 0       & 0       & K_{zz}  \end{array} \right].
  \label{conddiag}
\end{align}


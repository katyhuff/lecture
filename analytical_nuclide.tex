\section{Solute Transport in Porous Media}

\subsection{Derivation \cite{van_genuchten_analytical_1982, 
leij_analytical_1991}}

Here we shall derive the general equation for solute transport in heterogeneous 
porous media.

As discussed in previous sections, mass transport by advection is driven by bulk 
water velocity, diffusion is the result of random thermal motion and tends  to 
drive solutes down concentration gradients, and hydraulic dispersion
results from heterogeneities in the water velocity field and drives accelerated 
mixing at the head of advection fronts.

Fundamentally, the effect of these phenomena on mass transport is captured by 
the conceptual mass conservation expression \begin{align}
  \mbox{In} - \mbox{Out} &= \mbox{Change in Storage}
  \label{inout}
\end{align} which describes the mass balance within a reference volume of porous 
media.

By rearranging equation \ref{inout} and defining incoming and outflowing fluxes 
in a control  volume,  solute transport in a permeable medium of homogeneous 
porosity can be
written (as in Schwartz and Zhang \cite{schwartz_fundamentals_2003})

\begin{align} \frac{\partial n C}{\partial t} & = - \nabla \cdot  (F_c + F_{dc} 
  + F_d) + m \label{solperm}
  \intertext{where} \displaybreak[0]
  n &= \mbox{solute accessible porosity } [\%]\nonumber\\
  C &= \mbox{ concentration } [kg \cdot m^{-3}]\nonumber\\ t &= \mbox{ time } 
  [s]\nonumber\\ F_c &= \mbox{ advective flow } [kg \cdot m^{-2}\cdot 
  s^{-1}]\nonumber\\
  &= nvC \nonumber \\
  \displaybreak[0]
  F_{dc} &= \mbox{ dispersive flow } [kg \cdot m^{-2}\cdot s^{-1}]\nonumber\\ &= 
  \alpha nv \nabla C  \nonumber\\ \displaybreak[0]
  F_d &= \mbox{ diffusive flow } [kg \cdot m^{-2}\cdot s^{-1}]\nonumber\\
  &= nD_e \nabla C\nonumber\\
  \displaybreak[0]
  m &= \mbox{ solute source } [kg \cdot m^{-3}\cdot s^{-1}].\nonumber
  \displaybreak[0]
  \intertext{In the expressions above,} v &= \mbox{ advective velocity } [m\cdot 
  s^{-1}] \nonumber\\
  \alpha &= \mbox{ dispersivity } [m]\nonumber\\
  D_e &= \mbox{ effective diffusion coefficient } [m^2\cdot s^{-1}]\nonumber
  \displaybreak[0]
  \intertext{and} n\cdot v &= \mbox{ Darcy flux } [m\cdot s^{-1}].
\end{align} 

The method by which the dominant solute transport mode (diffusive or advective)
is determined for a particular porous medium is by use of the dimensionless
Peclet number, 

\begin{align} Pe &= \frac{nvL}{\alpha nv + D_e},\\
  &= \frac{\mbox{advective rate}}{\mbox{diffusive rate}}\nonumber
  \intertext{where} L &= \mbox{ transport distance } [m].\nonumber
\end{align}

For a high $Pe$ number, advection is the dominant transport mode, while 
diffusive or dispersive transport dominates for a low $Pe$ number. If one of 
these terms can be neglected, the solution is simplified. 

Otherwise, the analytical expression in equation \eqref{solperm} will be the 
foundation of simplification by regression analyses for the radionuclide 
transport interface between components of the repository system model 
representing permeable porous media.  

It is customary to define the combination of molecular diffusion and mechanical
mixing as the dispersion tensor, $D$, such that the mass conservation equation 
becomes:

\begin{align}
  \nabla \left( nD\nabla C \right) - \nabla \left( nv \right) &= 
  \frac{\partial(nC)}{\partial t}
  \label{massbal} \intertext{Adding sorption, by accounting for a change in mass 
  storage,}
  \nabla \left( nD\nabla C \right) - \nabla \left( nv \right)  &= 
  \frac{\partial(nC)}{\partial t}  + \frac{\partial(s\rho_b)}{\partial t} 
  \label{withsorption} \intertext{where}
  s &= \mbox{sorbed phase concentration}\nonumber\\
  \rho_b &= \mbox{ bulk (dry) density }[kg/m^3].\nonumber
\end{align}

If it is assumed that sorption can be approximated as a linear isotherm, 
reversible reaction,

\begin{align}
  \nabla \left( nD\nabla C \right) - \nabla \left( nv \right)  &= 
  \frac{\partial(nC)}{\partial t}  + \frac{\partial(s\rho_b)}{\partial t} 
  \label{linisomasstrans}
  \intertext{where}
  K_d &= \mbox{species distribution coefficient.}\nonumber\\
\end{align}

This becomes 

\begin{align}
  \nabla \left( nD\nabla C \right) - \nabla \left( nv \right)  &= 
  \frac{\partial(nC)}{\partial t}  + \frac{\partial(K_dC\rho_b)}{\partial t} 
  \intertext{which, rearranged, gives}
  \nabla \left( nD\nabla C \right) - \nabla \left( nv \right)  &= 
  \frac{\partial}{\partial t}\left(nC + K_dC\rho_b\right)\\
  \nabla \left( nD\nabla C \right) - \nabla \left( nv \right)  &= 
  \frac{\partial}{\partial t}\left(nC\left(1 + 
  \frac{K_d\rho_b}{n}\right)\right).
  \label{sorptionrearranged}
\end{align}

In equations \eqref{sorptionrearranged} it is clear that the storage term can be 
simplified with a retardation factor, such that if

\begin{align}
  R_f &= \mbox{retardation factor}\\
  &= 1+\frac{\rho_bK_d}{n}\\
  \intertext{then equation \eqref{sorption rearranged} can be written}
  \nabla \left( nD\nabla C \right) - \nabla \left( nv \right) &= 
  R_f\frac{\partial(nC)}{\partial t}    \label{withlinsorption}
\end{align}

For uniform flow, the dispersion tensor, $D$, in equation \ref{uniflow} gives

\begin{align}
  D_x \frac{\partial^2 C}{\partial x^2} +
  D_y \frac{\partial^2 C}{\partial y^2} +
  D_z \frac{\partial^2 C}{\partial z^2} +
  v_x \frac{\partial C}{\partial x}  = R_f \frac{\partial C}{\partial t}.  
  \label{unidirflow}
\end{align}

A special case of uniform flow, no flow, simplifies to the diffusion equation,
\begin{align}
  D_x \frac{\partial^2 C}{\partial x^2} +
  D_y \frac{\partial^2 C}{\partial y^2} +
  D_z \frac{\partial^2 C}{\partial z^2}  = R_f \frac{\partial C}{\partial t} .
  \label{diffusion}
\end{align}

Solutions to these equations can be categorized by their boundary conditions.  
The first, specified-head or Dirichlet type boundary conditions define a specified species 
concentration on some section of the boundary of the representative volume, 

\begin{align}
  C(x,y,z,t) = C_0(x,y,z,t)\hspace{1mm}\mbox{ for } \left( x,y,z \right) \in 
  \Gamma.
\end{align}

The second type, specified-flow or Neumann type boundary conditions describe a full set of 
concentration gradients at the boundary of the domain

\begin{align}
  \frac{\partial C(\vec{r},t)}{\partial r} &= nD\vec{J} \hspace{1mm}\mbox{ for } 
  \vec{r} \in \Gamma.
  \intertext{where}
  \vec{r} &= \mbox{ position vector }\nonumber\\
  \Gamma &= \mbox{ domain boundary }\nonumber\\
  \vec{J} &= \mbox{ solute mass flux } [kg/m^2\cdot s].\nonumber
\end{align}

The third, head-dependent mixed boundary condition or Cauchy type, defines a solute 
flux along a boundary,

\begin{align}
  vC(X,y,z,t) - D_x \frac{\partial C(x,y,z,t}{\partial x} &= 
  vg(x,y,z,t)\hspace{1mm}\mbox{ for } \left( x,y,z \right) \in \Gamma
  \intertext{where}
  g(x,y,z,t) &= \mbox{ arbirary flux profile}
  \intertext{this can also be written}
  -nD_{ij}\frac{\partial C}{\partial x_j}\hat{i} + q_iC\hat{i} &= q_iC_0\hat{i}.
  \intertext{where}
  D_{i,j} &= \hat{i}\mbox{ component of the } [m^2/s]\nonumber\\
  q_i\hat{i} &= \mbox{outward fluid flux} [m/s]\nonumber\\
  \hat{i} &= \mbox{unit vector normal to the surface} [-]\nonumber\\
  C_0 &= \mbox{concentration of the fluid at the boundary} [kg/m^3]\nonumber\\
\end{align}

\subsection{Useful Known Solutions}
Various solutions to the advection dispersion equation in Equation 
\eqref{unidirflow} have been published for both the first and third types of 
boundary conditions. The third, Cauchy type, is mass conservative, and will be 
the primary kind of boundary condition used at the source for this model.


\subsubsection{One Dimensional Semi-Infinite Solution}
The conceptual model in Figure \ref{fig:1dinf} represents solute transport
in one dimension with unidirectional flow and a semi-infinite boundary condition 
in the positive flow direction. 

\vspace{1cm}
\begin{figure}[htbp!]
  \begin{center}
    \def\svgwidth{.5\textwidth}
    \input{1dinf.eps_tex}
  \end{center}
  \caption{Case I, a one dimensional, semi-infinite model.}
  \label{fig:1dinf}
\end{figure}

With the boundary conditions
\begin{align}
  -D \frac{\partial C}{\partial x}\big|_{x=0} + v_xc &= \begin{cases}
    vC_0  &  \left( 0<t<t_0 \right)\\
    0  &  \left( t>t_0 \right)\\
  \end{cases}\\
  \frac{\partial C}{\partial x}\big|_{x=\infty} &= 0
  \intertext{and the initial condition}
  C(x,0) &= C_i,
  \label{1dinfBC}
  \intertext{the solution is given as }
  C(x,t) &=\begin{cases}
    C_i+\left( C_0 - C_i \right)A(x,t) & 0<t<t_0\\
    C_i+\left( C_0 - C_i \right)A(x,t) - C_0A(x,t-t_0) & t>t_0
  \end{cases}
  \intertext{where}
  A(x,t) &= \frac{1}{2}\erfc{ \left[\frac{Rx-vt}{2\sqrt{DRt}}\right] } + \sqrt{ 
  \frac{v^2t}{\pi DR} }e^{-\frac{\left( Rx-vt \right)^2}{4DRt}}.
\end{align}


\subsubsection{One Dimensional Semi-Infinite Solution with Discrete Source}
The conceptual model in Figure \ref{fig:1dinf} represents solute transport
in one dimension with unidirectional flow and a semi-infinite boundary condition 
in the positive flow direction. 

\vspace{1cm}
\begin{figure}[htbp!]
  \begin{center}
    \def\svgwidth{.5\textwidth}
    \input{1dinfwithsrc.eps_tex}
  \end{center}
  \caption{Case II, a one dimensional, semi-infinite model.}
  \label{fig:1dinf}
\end{figure}

With the boundary conditions
\begin{align}
  -D \frac{\partial C}{\partial x}\big|_{x=0} + v_xc &= \begin{cases}
    vC_0  &  \left( 0<t<t_0 \right)\\
    0  &  \left( t>t_0 \right)\\
  \end{cases}\\
  \frac{\partial C}{\partial x}\big|_{x=\infty} &= 0
  \intertext{and the initial condition}
  C(x,0) &= C_i,
  \label{1dinfBC}
  \intertext{the solution is given as }
  C(x,t) &=\begin{cases}
    C_i+\left( C_0 - C_i \right)A(x,t) & 0<t<t_0\\
    C_i+\left( C_0 - C_i \right)A(x,t) - C_0A(x,t-t_0) & t>t_0
  \end{cases}
  \intertext{where}
  A(x,t) &= \frac{1}{2}\erfc{ \left[\frac{Rx-vt}{2\sqrt{DRt}}\right] } + \sqrt{ 
  \frac{v^2t}{\pi DR} }e^{-\frac{\left( Rx-vt \right)^2}{4DRt}}.
\end{align}

\subsubsection{One Dimensional Finite Solution}
The conceptual model in Figure \ref{fig:1dfin} represents solute transport
in one dimension with unidirectional flow and a finite boundary condition in the 
positive flow direction. 

\vspace{1cm}
\begin{figure}[htbp!]
  \begin{center}
    \def\svgwidth{.5\textwidth}
    \input{1dfin.eps_tex}
  \end{center}
  \caption{Case III, a one dimensional, finite model.}
  \label{fig:1dfin}
\end{figure}

With the boundary conditions
\begin{align}
  -D \frac{\partial C}{\partial x}\big|_{x=0} + v_xc &= \begin{cases}
    vC_0  &  \left( 0<t<t_0 \right)\\
    0  &  \left( t>t_0 \right)\\
  \end{cases}\\
  \frac{\partial C}{\partial x}\big|_{x=L} &= 0
  \intertext{and the initial condition}
  C(x,0) &= C_i,
  \label{1dinfBC}
  \intertext{the solution is given as }
  C(x,t) &=\begin{cases}
    C_2+\left( C_1 - C_2 \right)A(x,t) + \left( C_0-C_1 \right)B(x,t)& 0<t<t_0\\
    C_2+\left( C_1 - C_2 \right)A(x,t) + \left( C_0-C_1 \right)B(x,t) - 
    C_0B(x,t-t_0) & 0<t<t_0\\
  \end{cases}
  \intertext{where}
  A(x,t) &= \frac{1}{2}\erfc{\left[ \frac{R(x-x_1)-vt}{2\sqrt{DRt}} \right] } + 
  \sqrt{\frac{v^2t}{\pi DR}} e^{\left[\frac{vx}{D} - \frac{R}{4Dt}(x+x_1 + 
  \frac{vt}{R})^2\right]}\nonumber\\
  & - \frac{1}{2}\left[ 1+ \frac{v(x+x_1)}{D} + 
  \frac{v^2t}{DR}\right]e^{\frac{vx}{D}} \erfc{\left[ 
  \frac{R(x+x_1)+vt}{2\sqrt{DRt}} \right]}\\ B(x,t) &= \frac{1}{2}\erfc{\left[ 
  \frac{Rx-vt}{2\sqrt{DRt}} \right]} + \sqrt{\frac{v^2t}{\pi DR}}e^{-\left[ 
  \frac{(Rx-vt)^2}{4DRt} \right]}\nonumber\\
  & - \frac{1}{2}\left[ 1+ \frac{vx}{D} + \frac{v^2t}{DR}\right]e^{\frac{vx}{D}} 
  \erfc{\left[ \frac{Rx+vt}{2\sqrt{DRt}} \right]}.  \end{align}


\subsubsection{Three Dimensional Semi-Infinite Solution}
The conceptual model in Figure \ref{fig:3dinf} represents solute transport
in three dimensions with unidirectional flow and a semi-infinite boundary 
condition in the positive flow direction.  

\begin{figure}[htbp!]
  \begin{center}
    \def\svgwidth{.8\textwidth}
    \input{3dinf.eps_tex}
  \end{center}
  \caption{Case IV, a three dimensional, semi-infinite model. Figure from 
  \cite{leij_analytical_1991}.}
  \label{fig:3dinf}
\end{figure}

With the boundary conditions
\begin{align}
  -D \frac{\partial C}{\partial x}\big|_{x=0} + v_xc &= \begin{cases}
    vC_0  &  \left( 0<t<t_0 \right)\\
    0  &  \left( t>t_0 \right)\\
  \end{cases}\\
  \frac{\partial C}{\partial x}\big|_{x=L} &= 0
  \intertext{and the initial condition}
  C(x,0) &= C_i,
  \label{1dinfBC}
  \intertext{the solution is given as }
  C(x,t) &=
    C_0\int_0^a \Lambda_6(t)\Xi(\rho,t)d\rho  + 
    \frac{\lambda}{2R}\int_0^t\int_0^\infty\Xi(\rho,\tau)\Lambda_4(\tau)d\rho 
    d\tau\\
  \intertext{where}
  \Xi(\rho,\tau) &= \frac{\rho R}{4D_r\tau}e^{\left(-  \frac{R(r^2 + 
  \rho^2)}{4D_r\tau} \right)}I_0\left( \frac{Rr\rho}{2D_r\tau} \right),\\
  I_0(x) &= \sum_{m=0}^\infty \frac{1}{(m!)^2}\left( \frac{x}{2}\right)^{2m}\\
  \Lambda_4(\tau) &= \erfc{\left[ \frac{v\tau-Rx}{\sqrt{4RD_xt}} \right]}  + 
  \left( 1+ \frac{v}{D_x}(x+v\tau/R \right)
   e^{\left( \frac{vx}{D_x} \right)}
  \erfc{\left[ \frac{Rx +v\tau}{\sqrt{4RD_xt}} \right]} \nonumber\\ & - \sqrt{ 
  \frac{4v^2\tau}{\pi R D_x} } e^{-\left( \frac{(Rx - 
  v\tau)^2}{\sqrt{4RD_x\tau}} \right)}, \intertext{and}
  \Lambda_6(t) &= e^{\frac{vx}{D_x}}\Bigg[ \left( 1+ \frac{v}{D_x}(x + x_1 + 
  vt/R) \right)
    \erfc{\left[ \frac{R(x+x_1)+vt}{\sqrt{4RD_xt}} \right]}      \nonumber\\
   & \hspace{1cm}- \left( 1+ \frac{v}{D_x}(x_1+x_2)+vt/R \right)
    \erfc{\left[ \frac{R(x+x_2)+vt}{\sqrt{4RD_xt}} \right]} \Bigg]\nonumber\\
   & \hspace{1cm}+ \erfc{\left[ \frac{R(x-x_2)-vt}{\sqrt{4RD_xt}} \right]i} - 
   \erfc{\left[ \frac{R(x-x_1)-vt}{\sqrt{4RD_xt}} \right]}     \nonumber\\
   & \hspace{1cm}+ \sqrt{ \frac{4v^2t}{\pi R D_x} }e^{\left( \frac{vx}{D_x} 
   \right)}\left[ e^{-\left( \frac{[R(x+x_2) + vt]^2}{4RD_xt} \right)}
     - e^{-\left( \frac{[R(x+x_1) + vt]^2}{4RD_xt} \right)}
   \right].  \end{align}

Clearly, this functional form of the solution could be onerous to code. Thus, 
other solutions are being considered, including the lumped parameter model of 
radionuclide transport. 

\subsubsection{Lumped Parameter Model}

For systems in which the flow can be assumed constant, it is possible to model a 
system of volumes as a connected lumped paramter models (Figure 
\ref{fig:lumpedseries}). 


\begin{figure}[htbp!]
  \begin{center}
    \def\svgwidth{.8\textwidth}
    \input{lumpedseries.eps_tex}
  \end{center}
  \caption{A system of volumes can be modeled as lumped parameter models in 
  series.}
  \label{fig:lumpedseries}
\end{figure}

The method by which each lumped parameter component is modeled is according to a 
relationship between the incoming concentration, $C_{in}(t)$, and the outgoing 
concentration, $C_{out}(t)$, \begin{align}
  C_{out}(t) &= \int_{-\infty}^t C_{in}(t')g(t-t')e^{-\lambda(t-t')}dt'
  \label{lumped1}
  \intertext{equivalently}
  C_{out}(t) &= \int_0^\infty C_{in}(t-t')g(t')e^{-\lambda t'}dt'
  \label{lumped2}
  \intertext{where}
  t'  &= \mbox{ time of entry }[s]\nonumber\\
  t-t'  &= \mbox{ transit time }[s]\nonumber\\
  g(t-t')  &= \mbox{ response function, a.k.a. transit time 
  distribution}[-]\nonumber]\\
  \lambda &= \mbox{ radioactive decay constant, 1 to neglect}[s^{-1}]
\end{align}

Selection of the response function is usually based on experimental tracer 
results in the medium at hand. However, some functions used commonly in chemical 
engineering applications include the Piston Flow Model (PFM), \begin{align}
  g(t') &= \delta{(t'-t_t))}
  \intertext{ the Exponential Model (EM) }
  g(t') &= \frac{1}{t_t}e^{-\frac{t'}{t_t}}
  \intertext{ and the Dispersion Model (DM), }
  g(t') &= \frac{\left[ \left( \frac{t\pi t'}{t_t \emph{Pe}} \right) 
  (\frac{1}{t'})e^{- \left( 1- \frac{t'}{t_t}  \right)^2} 
  \right]}{\frac{4t'}{t_t\emph{Pe}}}, \intertext{where}
  \emph{Pe}  &= \mbox{ Peclet number }[-]\nonumber\\
  t_t  &= \mbox{ mean tracer age }[s]\nonumber\\
    &= t_w \mbox{ if there are no stagnant areas}\nonumber\\
  t_w  &= \mbox{ mean residence time of water}[s]\nonumber\\
       &= \frac{V_m}{Q}\nonumber\\
       &= \frac{x}{v_w}\nonumber\\
       &= \frac{xn_e}{v_f}\nonumber
  \intertext{in which}
  V_m  &= \mbox{ mobile water volume}[m^3]\nonumber\\
  Q &= t_w \mbox{ volumetric flow rate }[m^3/s\nonumber\\
  v_w  &= \mbox{ mean water velocity}[m/s]\nonumber\\
  v_f  &= \mbox{ Darcy Flux}[m/s]\nonumber
  \intertext{and}
  n_e  &= \mbox{ effective porosity}[\%]\nonumber.
\end{align}
The latter of these, the Dispersion Model satisfies the one dimensional 
advection-dispersion equation, and is therefore the most physically relevant for 
this application. 

The solutions to these for constant concentration at the source boundary give
\begin{align}
  C(t) &=\begin{cases}
    PFM & C_0e^{\lambda t_t}\\
    EM  & \frac{C_0}{1+\lambda t_t}\\
    DM & \frac{C_0e^{\emph{Pe}\sqrt{\left( 1-(1+\frac{4\lambda 
    t_t}{\emph{Pe}})\right)}}}{2}\\
  \end{cases}
  \label{lumpedsolns}
\end{align}

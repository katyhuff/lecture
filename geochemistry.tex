\section{Geochemistry}

Geochemical phenomena significantly affect solute transport. As such, the 
geochemical attributes of various environments contribute differently to 
repository performance.

\subsection{Saturated and Unsaturated Environments}

Saturated repository concepts are found below the water table. Water permeates 
the rock.

Unsaturated repository concepts are found above or near the top of the water 
table. Water does not permeate the rock.

Current investigations are focused primarily on saturated environments.

\subsection{Reducing and Oxidizing Environments\cite{mason_oxidation_1949}}

Elements exist in various oxidation states, a positive or negative number  
refering to the charge of the atom due to the number of electrons it posesses.
Oxidation refers to the increase of the oxidation state number of an element, 
and reduction is the opposite.  That is, while oxidation is the chemical loss of 
electrons by an atom, reduction is the gaining of electrons by an atom. 

The redox potential is the tendency in potential energy for an element to give 
up its electron and become more oxidized. This  potential energy is measured 
relative to the potential for hydrogen atoms at unit concentration to lose 
electrons and become oxidized, in volts,
\begin{align}
  H_2 \rightarrow 2H^+  + 2e.
  \label{hydrogen}
\end{align}

The potential for the redox equation in equation \eqref{hydrogen} is chosen to 
be 0 volts, such that for positive values of the redox potential, reactions tend 
to proceed in the oxidation direction. Accordingly, reactions proceed in
the reduction direction for negative potentials.

Closed, saturated envrionments are reducing. Unsaturated environments and 
environments that are saturated and open are oxidizing. 

The far field of most reducing environments are slightly oxidizing as they reach 
the zone near the overlying aquifer.

\subsection{Salinity}

Salinity affects sorption and solubility behaviors. It also expedites corrosion.  
Finally, salt is an important indicator of the historical flow at a site.  Where 
should salinity information get incorporated in this model?

\subsection{Sorption}

Sorption is a factor that retards movement, in which solutes are incorporated 
into the surfaces of pores or fractures (adsorption) or incorporated into the 
rock matrix (absorption). More specificially, when water with some solute 
concentration, $C_i$, meets a porous medium or a granular solid, the reaction 
proceeds toward an equilibrium in which the solute mass has partitioned between 
the solution and the solid, such that,
\begin{align}
  S&=\frac{V_w(C_i -C)}{m_s}
  \intertext{where}
  S &= \mbox{ mass sorbed on the surface }[kg/kg]\nonumber\\
  V_w &= \mbox{ volume of the solution  }[m^3]\nonumber\\
  C_i &= \mbox{ initial concentration in the solution }[kg_m^3]\nonumber\\
  C &= \mbox{ equilibrium concentration in the solution }[kg/m^3]\nonumber\\
  m_s &= \mbox{ sediment mass }[kg].\nonumber
\end{align}
The function $S(C)$ is called a sorption isotherm. Two common forms for the 
sorption isotherm are
\begin{align}
  S &= \begin{cases}
    \text{Freundlich : } & KC^n\\ \text{Langmuir : } &\frac{Q^0KC}{1+KC}\\ 
  \end{cases}
  \label{isotherms}
  \intertext{where}
  K &= \mbox{ partition coefficient }[m^3/kg]\nonumber\\
  n &= \mbox{ shaping constant }[-]\nonumber
  \intertext{and}
  Q^0 &= \mbox{ maximum sorptive capacity }[kg/kg].\nonumber
  \intertext{If in the Freundlich isotherm,}
  n &=1, \intertext{then}
  S &= K_dC
  \intertext{where}
  K_d &= \mbox{ the distribution coefficient }[m^3/kg],\nonumber
\end{align}
and the isotherm becomes very easy to incorporate into the mass transport 
equations, as in equation \eqref{linisomasstrans}.

The simplest way to model sorption is with a linear isotherm and tabulated 
values.  Sorption coefficients such as $K_d$ are tabulated elementally because 
the character of the equilibrium reaction is element specific. For example, 
iodine has different sorption behavior than carbon. These coefficients are
also host rock specific. That is, each element has different sorption behaviors 
in clay than in granite. 

Geochemistry also plays a role, and sorption for some elements are higher in
reducing environments. Sorption for some elements may be higher in oxidizing
environments. 

\subsection{Solubility} 

The dissolution behavior of solutes in aqueous solutions is called its 
solubility. This behavior is limited by the solute's solubility limit, described  
by an equilibrium constant that depends upon temperature, water chemistry, and 
the properties of the element. The solubility constant for ordinary solutes, 
$K_s$ gives units of concentration, $[kg/m^3]$,   and can be determined 
algebraically by the law of mass action which gives the partitioning at 
equilibrium between reactants and products.  For a reaction
\begin{align}
  cC + dD &= yY + zZ,
  \intertext{where}
  c,d,y,z  &= \mbox{ amount of respective constituent }[mol]\nonumber\\
  C,D  &= \mbox{ reactants }[-]\nonumber\\
  Y,Z  &= \mbox{ products }[-]\nonumber,
  \intertext{the law of mass action gives}
  K &= \frac{(Y)^y(Z)^z}{(C)^c(D)^d}
  \intertext{where}
  (X)  &= \mbox{ the equilibrium molal concentration of X }[mol/m^3]\nonumber\\
  K  &= \mbox{ the equilibrium constant }[-].\nonumber
  \label{massaction}
\end{align}
The equillibrium constant for many reactions are known, and can be found in 
chemical tables. Thereafter, the solubility constraints of a solution at 
equilibrium can be found algebraically.  In cases of salts that  dissociate in 
aqueous solutions, this equilibrium constant is called the salt's solubility 
product $K_{sp}$.

This equillibrium model, however, is only appropriate for dilute situations, and 
nondilute solutions at  partial equilibrium must be treated with an activity 
model by substituting the activities of the constituents  for their molal 
concentrations,
\begin{align}
  [X] &= \gamma_x(X)
  \intertext{where}
  [X]  &= \mbox{ activity of X }[-]\nonumber\\
  \gamma_x  &= \mbox{ activity coefficient of X}[-]\nonumber\\
  (X)  &= \mbox{ molal concentration of X}[mol/m^3]\nonumber
  \intertext{such that}
  IAP &= \mbox{ Ion Activity Product }[-].\nonumber\\
      &= \frac{[Y]^y[Z]^z}{[C]^c[D]^d}\\
  \label{IAP}
\end{align}
The ration between the IAP and the equillibrium constant $(IAP/K)$ quantifiesn
the departure from equilibrium of a solution.  This information is useful during 
the transient stage in which a solute is first introduced to a solution. When 
$IAP/K<1$, the solution is undersaturated with respect to the products. When, 
conversely, $IAP/K>1$, the solution is oversaturated and precipitation of solids 
in the volume will occur. 

Two models of mass balance applicable to a control volume incorporating 
solubility
limitation include the Ahn and Hedin models.

\subsubsection{Ahn Solubility Limited Release Model}

In the Ahn models, radionuclides with lower solubility coefficients are modeled 
with
the solubility limited release model.  Solubility values are assumed from TSPA
for this model, and elements with a solubility of less than $~5\times 10^{-2}
[mol/m^3]$ are taken to be
`low.' Elements in this `low' category include Zr, Nb, Sn and some toxic 
actinides such as Th and Ra for an oxidizing, unsaturated environment similar to 
the Yucca Mountain Repository.
It should be noted that in a reducing environment, the actinides are not as 
mobile, and the high and low solubility radionuclides will differ from this 
model.
This model suggests that dissolution of radionuclides into the flowthrough water 
is dominated by diffusion, which is largely dependent upon the concentration 
gradient between the waste matrix and the water. The mass balance driving 
radionuclide release takes the form:

\begin{equation}
 \dot{m_i}=8\epsilon D_eS_iL\sqrt{\frac{Ur_0}{\pi D_e}}
\end{equation} 

where $\epsilon$, U, $r_0$, and L are the geometric and
hydrologic factors porosity, water velocity, waste package radius, and waste
package length, respectively. $D_e$ is the effective diffusion coefficient
($m^2/yr$)  and $S_i$ is the solubility ($kg/m^3$) of isotope $i$.


\subsubsection{Hedin Solubility Limited Release Model}

In the Hedin model of the waste matrix, the amount of solute available within
the waste package is solved for, and for radionuclides with low solubility, the 
mass
fraction released from the waste matrix is limited by a simplified description
of their solubility. That is, 

\begin{align} m_{1i}(t)\le v_{1i}(t)C_{sol}
\end{align}

where the mass $m_{1i}$ in $[kg]$ of a radionuclide $i$ dissolved into the waste 
package
void volume $v_1$ in $[m^3]$, at a time t, is limited by the solubility limit, 
the maximum concentration, $C_{sol}$ in $[kg/m^3]$ at which that radionuclide is 
soluble \cite{hedin_integrated_2002}.

Various things affect solubility. Temperature, salinity, etc.

A good model for solubility limitation is at the release boundary of the volume, 
for mixed volumes.


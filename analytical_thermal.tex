
\section{Analytical Models of Heat Transport} \label{sec:analytical_heat}
 
Heat transfer in saturated repository concepts will be dominated by conductive  
heat transfer. In a saturated closed system, very few air gaps will exist. For 
this reason,  heat transfer by radiation is likely to be negligible. Similarly, 
since  water velocities are comparatively low, heat transfer by mass transfer 
or by convection will be small relative to conduction.  

A discussion of analytical models of heat transport follows that quantifies these 
modes of heat transfer and addresses their applicability to the model at hand. 

%%%%%%%%%%%%%%%% Analytical Heat Transport %%%%%%%%%%%%%%%%%%%%%%%%

\subsection{Conduction}

Conductive heat transfer occurs as a result of a temperature gradient. Heat 
flows diffusively from the hotter material to the cooler material over time and
steadily approaches thermal equilibrium. The general form of the conduction 
equation can be expressed


\begin{align}
  \nabla^2T + \frac{q'''}{k} = \frac{1}{\alpha}\frac{\partial T}{\partial t},
  \label{general}
  \intertext{which with no heat source becomes the transient Fourier equation,}
  \nabla^2T  = \frac{1}{\alpha}\frac{\partial T}{\partial t},
  \label{transfourier}
  \intertext{or to the Laplace equation in steady state,}
  \nabla^2T = 0.
  \label{laplace}
  \intertext{Replacing the source gives the steady state Poisson equation,}
  \nabla^2T + \frac{q'''}{k} = 0.
  \label{poisson}
\end{align}

An areal heat flux, $q'' [W/m^{2}]$ can be derived from an integration of  
Poisson's equation \eqref{poisson}  and expressed in terms of the thermal 
conductivity of the material, $k [W/m\cdot^{\circ}K]$, and the
temperature gradient $\nabla T [K/m]$ by the expression

\begin{align}
  q''= -k\nabla T.
  \label{fourier}
\end{align}

For the one dimensional case, equation \ref{fourier} can be reduced using a 
finite difference approximation. For a body at $x_1$ with temperature $T_1$
and a body with temperature $T_2$ at position $x_2$,

\begin{align}
  q_x'' &= -k_x\frac{dT}{dx}\\
  &=-k_x\frac{(T_1-T_2)}{x_1-x_2}.
\end{align}

\subsection{Convection}

Convective heat transfer occurs advectively in accordance with fluid movement. 
Convection can be expressed as

\begin{align}
  \dot{q} &= -hA\Delta T
  \intertext{where}
  \dot{q} &= \mbox{ Thermal energy }[W] \nonumber\\
  h &= \mbox{ the heat transfer coefficient }[W\cdot m^{-2}]\nonumber\\
  A &= \mbox{ the surface area of heat transfer }[m^2]\nonumber
\end{align}

\subsection{Radiation}

Heat transfer by radiation is the result of the emission of electromagnetic 
waves. Planck black body radiation is analytically described, using $\sigma$, the   
Stefan-Boltzmann constant as

\begin{align}
  \dot{q} = \sigma A_1 F_{1\rightarrow 2}(T_1^4 - T_2^4)
  \label{planck}
  \intertext{where}
  \sigma =5.670373\times 10^{−8} [W/ m^{2} K^{4}]\nonumber
  \intertext{and}
  F_{1\rightarrow 2} =
  \begin{cases}
    \epsilon_1 &
    \mbox{ for a point source},\\
    \frac{1}{\frac{1}{\epsilon_1} + \frac{1}{\epsilon_2} - 1 } &
    \mbox{ for parallel plates},\\
    \frac{2\pi r_1 L}{\frac{1}{\epsilon_1} + \frac{1}{\epsilon_2} - 1 } \frac{r_1}{r_2} &
    \mbox{ for concentric cylinders}
  \end{cases}
  \intertext{where}
  \epsilon_i = \mbox{emissivity of surface i } [-]\nonumber\\
  r_i = \mbox{radius of cylinder i } [m] \nonumber\\
  L = \mbox{cylinder length }[m].\nonumber
\end{align}

\subsection{Mass Transfer}

Heat transfer by mass transfer is straightforward, resulting in the change in 
temperature in adjacent volumes as a result of matter movement. If the specific 
heat capacity of the transferred mass can be expressed as $c_p$, then the heat 
transfer is simply, 

\begin{align}
  \dot{q} = \dot{m}c_p\left( T_j - T_i \right)\\
  \intertext{where}
  c_p = \mbox{ specific heat capacity } [J/kg^{\circ}K].\nonumber
\end{align}


